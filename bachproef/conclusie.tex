%%=============================================================================
%% Conclusie
%%=============================================================================

\chapter{Conclusie}
\label{ch:conclusie}

% TODO: Trek een duidelijke conclusie, in de vorm van een antwoord op de
% onderzoeksvra(a)g(en). Wat was jouw bijdrage aan het onderzoeksdomein en
% hoe biedt dit meerwaarde aan het vakgebied/doelgroep? 
% Reflecteer kritisch over het resultaat. In Engelse teksten wordt deze sectie
% ``Discussion'' genoemd. Had je deze uitkomst verwacht? Zijn er zaken die nog
% niet duidelijk zijn?
% Heeft het onderzoek geleid tot nieuwe vragen die uitnodigen tot verder 
%onderzoek?

De opzet van dit onderzoek is om een antwoord te geven op de onderzoeksvragen “Welk dataformaat is performanter voor de Discovery API van de Fashion Society?”, “Is gRPC met ProtocolBuffers sneller dan REST API met JSON voor de implementatie van de Fashion Society?” en “Is gRPC met ProtocolBuffers efficiënter dan REST API met JSON op gebied van CPU gebruik voor de implementatie van de Fashion Society?”.
De conclusie is afhankelijk van de drie uitgevoerde testen, single call response time, totale response tijd voor vierduizend parallelle calls en totale response tijd voor tienduizend parallelle calls, die voor de methode GetService en GetServiceUrl zijn uitgevoerd.

Uit het onderzoek kunnen twee van de drie onderzoeksvragen beantwoord worden. Er kan geconcludeerd worden dat gRPC met ProtocolBuffers sneller is dan REST met JSON voor de implementatie van The Fasion Society en dat ProtocolBuffers performanter zijn dan JSON voor de implementatie van de Fashion Society.

Op de onderzoeksvraag “Is gRPC met ProtocolBuffers efficiënter dan REST API met JSON op gebied van CPU gebruik voor de implementatie van de Fashion Society?” kan geen antwoord gegeven worden omdat deze gegevens niet exact meetbaar waren gedurende het uitvoeren van de proof-of-concept. Hierbij is in overleg met de co-promotor het besluit gevormd dat de Fashion Society een snel groeiend bedrijf is waarbij performantie belangrijk is en dat als er meer CPU gebruik zou zijn dit een prijs is dat het bedrijf gewillig is om te betalen in ruil voor een performantere implementatie.






%%=============================================================================
%% Inleiding
%%=============================================================================

\chapter{\IfLanguageName{dutch}{Inleiding}{Introduction}}
\label{ch:inleiding}



\section{\IfLanguageName{dutch}{Probleemstelling}{Problem Statement}}
\label{sec:probleemstelling}

The Fashion Society is constant aan het uitbreiden, dit onder meer door overnames en het openen van eigen nieuwe ZEB winkels. Elk van deze nieuwe winkels werkt via het centraal systeem en maakt gebruik van de Discovery API bij ondermeer het informeren van de stock, verkopen van een artikel, etc. De Discovery API is een API die aan de hand van een binnenkomende call het adres van de gevraagde API gaat ophalen om aan te spreken. Naarmate er meer winkels zijn zullen er ook meer gelijktijdige requests afgehandeld worden door de Discovery API. Door deze toename aan requests kan de performantie dalen. Met dit onderzoek wordt hierop ingespeeld en wordt gekeken of gRPC met Protocol Buffers een performanter alternatief is op de huidige implementatie.

\section{\IfLanguageName{dutch}{Onderzoeksvraag}{Research question}}
\label{sec:onderzoeksvraag}

De onderzoeksvragen die gevormd worden bij de vergelijking tussen gRPC met Protocol Buffers en REST met JSON zijn als volgt.
\begin{itemize}
    \item Welk dataformaat is performanter voor de Discovery API van The Fashion Society?
    \item Is gRPC met ProtocolBuffers sneller dan REST API met JSON voor de implementatie van The Fashion Society?
    \item Is gRPC met ProtocolBuffers efficiënter dan REST API met JSON op gebied van CPU gebruik voor de implementatie van The Fashion Society?
\end{itemize}

\section{\IfLanguageName{dutch}{Onderzoeksdoelstelling}{Research objective}}
\label{sec:onderzoeksdoelstelling}

Aan de hand van een proof-of-concept wordt verwacht dat gRPC performanter zal zijn voor de Discovery API van The Fashion Society. Om deze doelstelling te behalen moet aan de onderzoeksvraag “Is gRPC met ProtocolBuffers sneller dan REST API met JSON voor de implementatie van The Fashion Society?” voldaan worden.

\section{\IfLanguageName{dutch}{Opzet van deze bachelorproef}{Structure of this bachelor thesis}}
\label{sec:opzet-bachelorproef}

% Het is gebruikelijk aan het einde van de inleiding een overzicht te
% geven van de opbouw van de rest van de tekst. Deze sectie bevat al een aanzet
% die je kan aanvullen/aanpassen in functie van je eigen tekst.

De rest van deze bachelorproef is als volgt opgebouwd:

In Hoofdstuk~\ref{ch:stand-van-zaken} wordt een overzicht gegeven van de stand van zaken binnen het onderzoeksdomein, op basis van een literatuurstudie.

In Hoofdstuk~\ref{ch:methodologie} wordt de methodologie toegelicht en worden de gebruikte onderzoekstechnieken besproken om een antwoord te kunnen formuleren op de onderzoeksvragen.

% TODO: Vul hier aan voor je eigen hoofstukken, één of twee zinnen per hoofdstuk

In Hoofdstuk~\ref{ch:conclusie}, tenslotte, wordt de conclusie gegeven en een antwoord geformuleerd op de onderzoeksvragen. Daarbij wordt ook een aanzet gegeven voor toekomstig onderzoek binnen dit domein.
%%=============================================================================
%% Inleiding
%%=============================================================================

\chapter{\IfLanguageName{dutch}{Inleiding}{Introduction}}
\label{ch:inleiding}

De inleiding moet de lezer net genoeg informatie verschaffen om het onderwerp te begrijpen en in te zien waarom de onderzoeksvraag de moeite waard is om te onderzoeken. In de inleiding ga je literatuurverwijzingen beperken, zodat de tekst vlot leesbaar blijft. Je kan de inleiding verder onderverdelen in secties als dit de tekst verduidelijkt. Zaken die aan bod kunnen komen in de inleiding~\autocite{Pollefliet2011}:

\begin{itemize}
  \item context, achtergrond
  \item afbakenen van het onderwerp
  \item verantwoording van het onderwerp, methodologie
  \item probleemstelling
  \item onderzoeksdoelstelling
  \item onderzoeksvraag
  \item \ldots
\end{itemize}

\section{\IfLanguageName{dutch}{Probleemstelling}{Problem Statement}}
\label{sec:probleemstelling}

Zoals eerder vermeld in deze bachelorproef is The Fashion Society constant aan het uitbreiden, dit onder meer door overnames en het openen van eigen nieuwe ZEB winkels. Elk van deze nieuwe winkels werkt via het centraal systeem en maakt dus onwetend gebruik van de Discovery API bij ondermeer het informeren van de stock, verkopen van een artikel, etc. Naarmate er meer winkels zijn zullen er ook meer gelijktijdige requests afgehandeld worden door de Discovery API. Door deze toename aan requests kan het al eens zijn dat de performantie daalt. Met dit onderzoek wordt hierop ingespeeld en wordt gekeken of gRPC met Protocol Buffers een performanter alternatief kan zijn op de huidige implementatie.

\section{\IfLanguageName{dutch}{Onderzoeksvraag}{Research question}}
\label{sec:onderzoeksvraag}

De onderzoeksvragen die gevormt worden bij de vergelijking tussen gRPC met Protocol Buffers en REST met JSON zullen dan ook zijn als volgt.
\begin{itemize}
    \item Welk dataformaat is het meest performant voor de Discovery API van The Fashion Society?
    \item Is gRPC met ProtocolBuffers sneller dan REST API met JSON voor de implementatie van The Fashion Society?
    \item Is gRPC met ProtocolBuffers efficiënter dan REST API met JSON voor de implementatie van The Fashion Society?
\end{itemize}

\section{\IfLanguageName{dutch}{Onderzoeksdoelstelling}{Research objective}}
\label{sec:onderzoeksdoelstelling}

Wat is het beoogde resultaat van je bachelorproef? Wat zijn de criteria voor succes? Beschrijf die zo concreet mogelijk. Gaat het bv. om een proof-of-concept, een prototype, een verslag met aanbevelingen, een vergelijkende studie, enz.

\section{\IfLanguageName{dutch}{Opzet van deze bachelorproef}{Structure of this bachelor thesis}}
\label{sec:opzet-bachelorproef}

% Het is gebruikelijk aan het einde van de inleiding een overzicht te
% geven van de opbouw van de rest van de tekst. Deze sectie bevat al een aanzet
% die je kan aanvullen/aanpassen in functie van je eigen tekst.

De rest van deze bachelorproef is als volgt opgebouwd:

In Hoofdstuk~\ref{ch:stand-van-zaken} wordt een overzicht gegeven van de stand van zaken binnen het onderzoeksdomein, op basis van een literatuurstudie.

In Hoofdstuk~\ref{ch:methodologie} wordt de methodologie toegelicht en worden de gebruikte onderzoekstechnieken besproken om een antwoord te kunnen formuleren op de onderzoeksvragen.

% TODO: Vul hier aan voor je eigen hoofstukken, één of twee zinnen per hoofdstuk

In Hoofdstuk~\ref{ch:conclusie}, tenslotte, wordt de conclusie gegeven en een antwoord geformuleerd op de onderzoeksvragen. Daarbij wordt ook een aanzet gegeven voor toekomstig onderzoek binnen dit domein.
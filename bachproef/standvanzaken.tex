\chapter{\IfLanguageName{dutch}{Stand van zaken}{State of the art}}
\label{ch:stand-van-zaken}

% Tip: Begin elk hoofdstuk met een paragraaf inleiding die beschrijft hoe
% dit hoofdstuk past binnen het geheel van de bachelorproef. Geef in het
% bijzonder aan wat de link is met het vorige en volgende hoofdstuk.

% Pas na deze inleidende paragraaf komt de eerste sectiehoofding.

In de inleiding is duidelijk geworden dat het onderzoek gericht zal zijn op twee mogelijke dataformaten aan de hand van 2 verschillende structuren, namelijk JSON aan de hand van REST en gRPC aan de hand van Protocol Buffers. Om dit onderzoek volledig te kunnen begrijpen is het belangrijk om de werking en de basisprincipes van deze dataformaten en de daarbij behorende structuren te begrijpen. Om deze reden zullen eerst de werking en basisprincipes van JSON en REST worden uitgelegd, en als volgt die van gRPC en Protocol Buffers.

\section{JSON}
\label{sec:JSON}

\subsection{Algemeen}
\label{subsec:Algemeen}

JavaScript Object Notation of in het kort, JSON, is een tekstsyntaxis of dataformaat dat geïnspireerd is door de object literalen van JavaScript dat ook gekend staat als ECMAScript. Het maakt een gegevensuitwisseling tussen alle programmeertalen op een gestructureerde manier mogelijk. Hiervoor maakt JSON gebruik van een structuur van accolades, haakjes, dubbele punten en komma's dewelke zeer nuttig kan zijn in verchillende contexten, profielen en applicaties. Ondanks dat JSON geïnspireerd is door ECMAScript probeert het de interne data representatie ervan niet op te leggen aan andere programmeertalen, in plaats daarvan deelt JSON een onderdeel van ECMAScript's syntax met alle andere programmeertalen. JSON mag niet gezien worden als een specificatie van een volledige gegevensuitwisseling want dat is het ook niet. Bij een zinvolle gegevensuitwisseling wordt er een overeenstemming over de semantiek die gekoppeld is aan een gebruik van de JSON-syntaxis tussen producent en consuiment vereist. JSON kan wel gezien worden als een syntactisch raamwerk waaraan een specifieke semantiek kan worden gekoppelt.

Doordat er veel verschillende types getallen zijn zoals, maar niet beperkt tot, decimale en binaire getallen, kiest JSON voor enkel een weergave van getallen die mensen gebruiken, namelijk een reeks cijfers. Ookal zijn alle programmeertalen het niet altijd eens over de interne representaties van getallen, ze weten wel hoe ze cijferreeksen moeten begrijpen.
Zoals nu wel duidelijk is kunnen programmeertalen sterk verschillen in mate van wat ondersteund wordt en hoe iets moet gerepresenteerd worden. Dit geldt ook voor objecten, niet alle programmeertalen ondersteunen objecten, en indien ze dit wel doen is er nog steeds een groot verschil in de kenmerken en beperkingen de objecten bieden. Met andere woorden kunnen de modellen van objectsystemen heel erg uiteenlopen. Om dit probleem aan te pakken biedt JSON een eenvoudige notatie aan voor het uitdrukken van verzamelingen met naam / waarde-paren. Om zulke verzamelingen weer te geven hebben de meeste programmeertalen reeds een functie zoals maar niet beperkt tot, struct, map, hash, object.
Daarnaast biedt JSON ook ondersteuning voor geordende zoeklijsten, alsook hiervoor hebben all programmeertalen een functie om deze weer te geven zoals array, vector en list. Aangezien objecten en arrays zich kunnen nesten kunnen aan de hand van JSON complexe structuren zoals boomstructuren worden gerepresenteerd.
Hieruit kan dus geconcludeerd worden dat door het aanvaarden van JSON's simpele conventies, complexe datastructuren uitgewisseld kunnen worden tussen wat anders incompatibele programmeertalen zijn.



\subsection{Representatie}
\label{subsec:Representatie}





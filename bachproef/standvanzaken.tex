\chapter{\IfLanguageName{dutch}{Stand van zaken}{State of the art}}
\label{ch:stand-van-zaken}

% Tip: Begin elk hoofdstuk met een paragraaf inleiding die beschrijft hoe
% dit hoofdstuk past binnen het geheel van de bachelorproef. Geef in het
% bijzonder aan wat de link is met het vorige en volgende hoofdstuk.

% Pas na deze inleidende paragraaf komt de eerste sectiehoofding.

In de inleiding is duidelijk geworden dat het onderzoek gericht zal zijn op twee mogelijke dataformaten aan de hand van 2 verschillende structuren, namelijk JSON aan de hand van REST en gRPC aan de hand van Protocol Buffers. Om dit onderzoek volledig te kunnen begrijpen is het belangrijk om de werking en de basisprincipes van deze dataformaten en de daarbij behorende structuren te begrijpen. Om deze reden zullen eerst de werking en basisprincipes van JSON en REST worden uitgelegd, en als volgt die van gRPC en Protocol Buffers.

\section{JSON}
\label{sec:JSON}



\lipsum[7-20]

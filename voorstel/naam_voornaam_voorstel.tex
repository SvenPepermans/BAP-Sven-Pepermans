%==============================================================================
% Sjabloon onderzoeksvoorstel bachelorproef
%==============================================================================
% Gebaseerd op LaTeX-sjabloon ‘Stylish Article’ (zie voorstel.cls)
% Auteur: Jens Buysse, Bert Van Vreckem
%
% Compileren in TeXstudio:
%
% - Zorg dat Biber de bibliografie compileert (en niet Biblatex)
%   Options > Configure > Build > Default Bibliography Tool: "txs:///biber"
% - F5 om te compileren en het resultaat te bekijken.
% - Als de bibliografie niet zichtbaar is, probeer dan F5 - F8 - F5
%   Met F8 compileer je de bibliografie apart.
%
% Als je JabRef gebruikt voor het bijhouden van de bibliografie, zorg dan
% dat je in ``biblatex''-modus opslaat: File > Switch to BibLaTeX mode.

\documentclass{voorstel}

\usepackage{lipsum}

%------------------------------------------------------------------------------
% Metadata over het voorstel
%------------------------------------------------------------------------------

%---------- Titel & auteur ----------------------------------------------------

% TODO: geef werktitel van je eigen voorstel op
\PaperTitle{Wat is de meest performante manier om data te ontvangen en te verzenden op het web: Een vergelijkende studie tussen gRPC met Protocol Buffers en REST met JSON en proof-of-concept }
\PaperType{Onderzoeksvoorstel Bachelorproef 2020-2021} % Type document

% TODO: vul je eigen naam in als auteur, geef ook je emailadres mee!
\Authors{Sven Pepermans\textsuperscript{1}} % Authors
\CoPromotor{
	Kristof van Moorter\textsuperscript{2}
 }
\affiliation{\textbf{Contact:}
  \textsuperscript{1} \href{mailto:sven.pepermans@student.hogent.be}{sven.pepermans@student.hogent.be};
  \textsuperscript{2} \href{mailto:kristofvanmoorter@zeb.be}{kristofvanmoorter@zeb.be};
}

%---------- Abstract ----------------------------------------------------------

\Abstract{
	In deze bachelorproef zal onderzoek gedaan worden naar de verschillen in performantie, tooling en usability tussen google Remote Procedure Call (gRPC) met Protocol Buffers en Representational State Transfer (REST) met JavaScript Object Notation (JSON). Eerst zal er onderzocht worden wat gRPC, Protocol Buffers, REST en JSON zijn en hoe deze werken. In het tweede deel wordt aan de hand van een proof-of-concept onderzocht welke manier van data verzenden en ontvangen het meest performant is en welke de betere usability heeft.
	
	De relevantie van dit onderzoek is te vinden in de digitalisering van bedrijven, alsmaar meer data wordt verzonden en ontvangen via het web aan de hand van webapplicaties zoals ASP.NET Core apps. Dit maakt de nood aan een performante manier van data te verzenden en ontvangen des te noodzakelijker. 
	Door het onderzoeken van het performantste dataformaat kan er binnen bedrijven bespaard worden op resources zoals processing power. Alsook zal een performant dataformaat de user experience verbeteren door de hogere doorvoersnelheid.
	Van deze studie wordt verwacht dat gRPC een hogere snelheid en een lager CPU gebruik zal hebben.
}

%---------- Onderzoeksdomein en sleutelwoorden --------------------------------
% TODO: Sleutelwoorden:
%
% Het eerste sleutelwoord beschrijft het onderzoeksdomein. Je kan kiezen uit
% deze lijst:
%
% - Mobiele applicatieontwikkeling
% - Webapplicatieontwikkeling
% - Applicatieontwikkeling (andere)
% - Systeembeheer
% - Netwerkbeheer
% - Mainframe
% - E-business
% - Databanken en big data
% - Machineleertechnieken en kunstmatige intelligentie
% - Andere (specifieer)
%
% De andere sleutelwoorden zijn vrij te kiezen

\Keywords{Webapplicatieontwikkeling. gRPC --- JSON --- API} % Keywords
\newcommand{\keywordname}{Sleutelwoorden} % Defines the keywords heading name

%---------- Titel, inhoud -----------------------------------------------------

\begin{document}

\flushbottom % Makes all text pages the same height
\maketitle % Print the title and abstract box
\tableofcontents % Print the contents section
\thispagestyle{empty} % Removes page numbering from the first page

%------------------------------------------------------------------------------
% Hoofdtekst
%------------------------------------------------------------------------------

% De hoofdtekst van het voorstel zit in een apart bestand, zodat het makkelijk
% kan opgenomen worden in de bijlagen van de bachelorproef zelf.
%---------- Inleiding ---------------------------------------------------------

\section{Introductie} % The \section*{} command stops section numbering
\label{sec:introductie}

Zonder dat we het weten maken wij, de internetgebruiker, continue gebruik van dataformaten en protocollen. Deze komen in verschillende soorten, XML, JSON, Protocol Buffers, GraphQL, ... Deze data formaten zorgen ervoor, aan de hand van een achterliggende implementatie in een Application Programming Interface (API), dat we onze vrienden hun nieuwe Facebook of Instagrampost kunnen zien in onze internetbrowser of op onze app. 
APIs maken het meest gebruik van XML en JSON, indien toch een dominerend formaat gekozen moet worden zal dit JSON zijn. JSON is veel sneller om te lezen en schrijven dan XML.

Echter wil dit niet zeggen dat JSON het snelste data formaat is. In deze Bachelorproef zal ik onderzoeken of Protocol Buffers geïmplementeerd aan de hand van gRPC eventueel een even performant of zelfs performanter alternatief kan zijn als REST API dat JSON gebruikt. De bijhorende onderzoeksvragen zullen dan ook als volgt zijn:


\begin{itemize}
  \item Is gRPC met ProtocolBuffers sneller dan een REST API met JSON?
  \item Is gRPC met ProtocolBuffers efficiënter dan een REST API met JSON?
\end{itemize}

%---------- Stand van zaken ---------------------------------------------------

\section{State-of-the-art}
\label{sec:state-of-the-art}
gRPC is open source Remote Procedure Call (RPC) framework en is in 2015 ontstaan uit zijn voorganger Stubby, deze was een single general-purpose RPC infrastructuur~\autocite{gRPC}. Deze technologie laat het toe voor een programma om procedures te starten op andere computers in, eventueel, verschillende adresruimten aan de hand van een smal comminucatiekanaal~\autocite{Nelson1981}. gRPC zelf is geen dataformaat zoals JSON maar kan gebruik maken van verschillende data formaten, standaard is dit Protocol buffers, ook wel Protobufs genoemd.

JSON en Protocol Buffers hebben zowel gelijkenissen als grote verschillen, beide zijn een language-neutral dataformaat zo blijkt uit de documentatie van Protocol Buffers~\autocite{Google} en  JSON~\autocite{Json2017}. Het grootste, visuele verschil voor de gebruiker is te vinden in de structuur. JSON is een collectie van naam/value paren of een geordende lijst van waarden, daarintegen maken Protobufs gebruik van een soort model, er moet maar eenmaal gedefinieerd worden hoe de data gestructureerd zal zijn, daarna kan gebruik gemaakt worden van gegenereerde code om gemakkelijk data van en naar een variëteit van data streams te lezen en schrijven. Dit kan allemaal geprogrammeerd worden in heel wat verschillende programmeertalen.

gRPC wordt reeds gebruikt door verschillende grote bedrijven zoals Cisco en Netflix. 
Cisco gebruikt gRPC als het ideale, uniforme transportprotocol voor modelgestuurde configuratie en telemetrie. Dit komt dankzij de ondersteuning voor hoogwaardige bidirectionele streaming, op TLS gebaseerde beveiliging en het brede scala aan programmeertalen. 
Netflix koos dan weer voor gRPC omdat ze belang hechtte aan de architectonische kennis in de IDL (proto) dat een onderdeel is van gRPC en aan de, van deze proto-afgeleide, codegeneratie. Daarnaast speelde ook de cross-language compatibility en codegeneratie in gRPC een belangrijke rol bij de keuze voor gRPC bij Netflix~\autocite{Foundation2018}.

Eerder werd er nog geen officieel onderzoek uitgevoerd naar dit onderwerp, echter zijn wel online artikels te vinden die gRPC met REST gaan vergelijken aan de hand van een benchmark. Eén zo een onderzoek is uitgevoerd door~\textcite{Fernando2019}. Dit onderzoek was een benchmark tussen gRPC met Protocol Buffers en REST met JSON concludeerd dat gRPC sneller was dan REST behalve bij het streamen van data, hier is gRPC iets trager als REST.

Het onderzoek dat in deze Bachelorproef zal uitgevoerd worden is gebaseerd op dat van Ruwan Fernando, hiermee wordt bedoeld dat deze proof-of-concept geprogrammeerd zal worden in C\# en dat de  verschillende impelementaties met elkaar vergeleken zullen worden aan de hand van een benchmark. Het verschil met Fernando R. zijn onderzoek kan men vinden in de verwerking van de data, in dit onderzoek zal de data uit een aanhangende databank gehaald worden. Alsook zal in dit onderzoek een tienvoud van het aantal iteraties doen.

% Voor literatuurverwijzingen zijn er twee belangrijke commando's:
% \autocite{KEY} => (Auteur, jaartal) Gebruik dit als de naam van de auteur
%   geen onderdeel is van de zin.
% \textcite{KEY} => Auteur (jaartal)  Gebruik dit als de auteursnaam wel een
%   functie heeft in de zin (bv. ``Uit onderzoek door Doll & Hill (1954) bleek
%   ...'')

%---------- Methodologie ------------------------------------------------------
\section{Methodologie}
\label{sec:methodologie}
Het onderzoek zal gevoerd worden aan de hand van simulaties en experimenten in een Proof-of-Content (PoC). Er zullen 2 identieke APIs opgesteld worden in C\#, een .NET Core REST API voor JSON, en .NET Core API voor gRPC. Aan de hand van een benchmark tool, dewelke nog niet specifiek gekozen is, zullen er 2 verschillende categorieën aan GET en POST calls uitgevoerd worden. Big payload calls, hier zal substantief meer data opgehaald en verzonden worden dan bij de Small payload calls. Daarnaast zal het experiment twee maal uitgevoerd worden, een eerste keer met 1000 iteraties en een tweede maal met 2000 iteraties, dit om inaccurate metingen van kleine uitvoertijden te verminderen.
Eens de resultaten van de experimenten verzameld zijn zullen we deze vergelijken en conclusies trekken.
%---------- Verwachte resultaten ----------------------------------------------
\section{Verwachte resultaten}
\label{sec:verwachte_resultaten}
Ik verwacht bij de grote payloads een duidelijk verschil in voordeel van gRPC en Protocol Buffers, echter verwacht ik wel dat REST API met JSON in de kleine payloads nog degelijk zal presteren en misschien zelf nog licht overwegend beter zal zijn dan gRPC en Protocol Buffers.

	\centering
	\includegraphics[width=1\linewidth]{screenshot001}



	\centering
	\includegraphics[width=1\linewidth]{screenshot002}

	



%---------- Verwachte conclusies ----------------------------------------------
\section{Verwachte conclusies}
\label{sec:verwachte_conclusies}

Dit onderzoek moet doen blijken of gRPC en Protocol Buffers sneller en efficiënter zijn dan een REST API met JSON. In de PoC zal duidelijk worden dat gRPC en Protocol Buffers een sneller en efficiënter alternatief zijn voor JSON. Dit resultaat zou volgens mij verklaard worden doordat JSON een ouder dataformaat is en dus niet geoptimaliseerd voor de huidige technologieën terwijl gRPC zeer recent ontworpen is om optimaal met de huidige technologieën om te gaan.



%------------------------------------------------------------------------------
% Referentielijst
%------------------------------------------------------------------------------
% TODO: de gerefereerde werken moeten in BibTeX-bestand ``voorstel.bib''
% voorkomen. Gebruik JabRef om je bibliografie bij te houden en vergeet niet
% om compatibiliteit met Biber/BibLaTeX aan te zetten (File > Switch to
% BibLaTeX mode)

\phantomsection
\printbibliography[heading=bibintoc]

\end{document}

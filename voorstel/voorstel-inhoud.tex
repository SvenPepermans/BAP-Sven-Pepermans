%---------- Inleiding ---------------------------------------------------------

\section{Introductie} % The \section*{} command stops section numbering
\label{sec:introductie}

Hier introduceer je werk. Je hoeft hier nog niet te technisch te gaan.

Je beschrijft zeker:

\begin{itemize}
  \item de probleemstelling en context
  \item de motivatie en relevantie voor het onderzoek
  \item de doelstelling en onderzoeksvraag/-vragen
\end{itemize}

%---------- Stand van zaken ---------------------------------------------------

\section{State-of-the-art}
\label{sec:state-of-the-art}
gRPC is open source Remote Procedure Call (RPC) framework en is in 2015 ontstaan uit zijn voorganger Stubby, deze was een single general-purpose RPC infrastructuur \autocite{Chalin} => (Patrice Chalin). Deze technologie laat het toe voor een programma om procedures te starten op andere computers in, eventueel, verschillende adresruimten aan de hand van een smal comminucatiekanaal. \autocite{Nelson1981}. gRPC zelf is geen dataformaat zoals JSON maar kan gebruik maken van verschillende data formaten, standaard is dit Protocol buffers, ook wel Protobufs genoemd.

JSON en Protocol Buffers hebben zowel gelijkenissen als grote verschillen, beide zijn een language-neutral dataformaat zo blijkt uit de documentatie van \autocite{Google} => Google en \autocite{Crockford} => Crockford. Het grootste, visuele voor de gebruiker is te vinden in de structuur. JSON is een collectie van naam/value paren of een geordende lijst van waarden, daarintegen maken Protobufs gebruik van een soort model, er moet maar eenmaal gedefinieerd worden hoe de data gestructureerd zal zijn, daarna kan gebruik gemaakt worden van gegenereerde code om gemakkelijk data van en naar een variëteit van data streams te lezen en schrijven. Dit kan allemaal geprogrammeerd worden in heel wat verschillende programmeertalen.

gRPC wordt reeds gebruikt door verschillende grote bedrijven zoals Cisco en Netflix. 
Cisco gebruikt gRPC als het ideale, uniforme transportprocol voor modelgestuurde configuratie en telemetrie. Dit komt dankzij de ondersteuning voor hoogwaardige bidirectionele streaming, op TLS gebaseerde beveiliging en het brede scala aan programmeertalen. 
Netflix aan de andere hand koos voor gRPC omdat ze belang hechtte aan de architectonische kennis in de IDL (proto) dat een onderdeel is van gRPC en aan de, van deze proto afgeleide, codegeneratie. Daarnaast speelde ook de cross-language compatibility en codegeneratie in gRPC een belangrijke rol bij de keuze voor gRPC bij Netflix \autocite{Foundation2018} => (Cloud Native Computing Foundation, 2018).

Eerder officiëel onderzoek is nog niet gevoerd naar dit onderwerp, echter zijn wel online artikels te vinden die gRPC met REST gaan vergelijken aan de hand van een benchmark. Eén zo een onderzoek is uitgevoerd door \textcite{Fernando2019} => Ruwan Fernando (2019). Dit onderzoek was een benchmark tussen gRPC met Protocol Buffers en REST met JSON concludeerd dat gRPC sneller was dan REST behalve bij het streamen van data, hier is gRPC iets trager als REST.

Het onderzoek dat in deze Bachelorproef zal uitgevoerd worden zal gelijkaardig zijn als dat van Ruwan Fernando, hiermee wordt bedoeld dat deze proof-of-concept geprogrammeerd zal worden in C#. Het verschil met Fernando R. zijn onderzoek kan men vinden in de verwerking van de data, in dit onderzoek zal de data uit een aanhangende databank gehaald worden. Alsook zal in dit onderzoek een tienvoud van het aantal iteraties doen.






Hier beschrijf je de \emph{state-of-the-art} rondom je gekozen onderzoeksdomein. Dit kan bijvoorbeeld een literatuurstudie zijn. Je mag de titel van deze sectie ook aanpassen (literatuurstudie, stand van zaken, enz.). Zijn er al gelijkaardige onderzoeken gevoerd? Wat concluderen ze? Wat is het verschil met jouw onderzoek? Wat is de relevantie met jouw onderzoek?

Verwijs bij elke introductie van een term of bewering over het domein naar de vakliteratuur, bijvoorbeeld~\autocite{Doll1954}! Denk zeker goed na welke werken je refereert en waarom.

% Voor literatuurverwijzingen zijn er twee belangrijke commando's:
% \autocite{KEY} => (Auteur, jaartal) Gebruik dit als de naam van de auteur
%   geen onderdeel is van de zin.
% \textcite{KEY} => Auteur (jaartal)  Gebruik dit als de auteursnaam wel een
%   functie heeft in de zin (bv. ``Uit onderzoek door Doll & Hill (1954) bleek
%   ...'')

Je mag gerust gebruik maken van subsecties in dit onderdeel.

%---------- Methodologie ------------------------------------------------------
\section{Methodologie}
\label{sec:methodologie}

Hier beschrijf je hoe je van plan bent het onderzoek te voeren. Welke onderzoekstechniek ga je toepassen om elk van je onderzoeksvragen te beantwoorden? Gebruik je hiervoor experimenten, vragenlijsten, simulaties? Je beschrijft ook al welke tools je denkt hiervoor te gebruiken of te ontwikkelen.

%---------- Verwachte resultaten ----------------------------------------------
\section{Verwachte resultaten}
\label{sec:verwachte_resultaten}

Hier beschrijf je welke resultaten je verwacht. Als je metingen en simulaties uitvoert, kan je hier al mock-ups maken van de grafieken samen met de verwachte conclusies. Benoem zeker al je assen en de stukken van de grafiek die je gaat gebruiken. Dit zorgt ervoor dat je concreet weet hoe je je data gaat moeten structureren.

%---------- Verwachte conclusies ----------------------------------------------
\section{Verwachte conclusies}
\label{sec:verwachte_conclusies}

Hier beschrijf je wat je verwacht uit je onderzoek, met de motivatie waarom. Het is \textbf{niet} erg indien uit je onderzoek andere resultaten en conclusies vloeien dan dat je hier beschrijft: het is dan juist interessant om te onderzoeken waarom jouw hypothesen niet overeenkomen met de resultaten.

